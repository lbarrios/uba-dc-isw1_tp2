\section{Casos de Uso}

\begin{figure}[H]
  \begin{center}
  \includegraphics[width=500px]{images/casos-de-uso.pdf}
  \end{center}
\end{figure}

\captionsetup[table]{name=Caso de uso}

%
% - CASO DE USO: 1) REGISTRANDOSE
%
\begin{casodeuso}
  \cutitle{Registrándose}
  \cuactors{Cliente}
  \cupre{-}
  \cupost{Se ha enviado un mail de bienvenida, el usuario se encuentra pendiente de confirmar.}
  \cucourse{
    1. El cliente ingresa su usuario y contraseña & \\
    2. El sistema valida que el usuario no exista. & 2.1 Si el usuario ya existe, mostrar mensaje e ir a 1. \\
    3. El sistema valida que la contraseña sea segura. & 3.1 Si la contraseña no es segura, informar al cliente e ir a 1 \\
    4. El cliente ingresa sus datos personales & \\
    5. Incluye CU \ref{cu:evaluado-estado-financiero}: evaluando estado financiero & 5.1 Si el cliente tiene deudas, denegarle el registro. FIN CU. \\
    6. El cliente ingresa su dirección & \\
    7. Incluye CU \ref{cu:validando-domicilio}: validando domicilio & 7.1 si el domicilio no es válido, mostrar mensaje de error e ir a 6. \\
    8. Si el cliente desea: es extendido por CU: Agregando datos de pago online & \\
    9. El cliente ingresa su mail & \\
    10. El sistema define el link y el contenido del mail de bienvenida & \\
    11. Incluye caso de uso \ref{cu:enviando-mail}: enviando mail & \\
    12. FIN CU & \\
  }
  \culabel{registrandose}
\end{casodeuso}

Este caso de uso responde al objetivo \textbf{lograr [cliente registrado] (1.1.1.1.1.1)}. 

Los pasos 1, 4, 6 y 9 se corresponden con el objetivo \textbf{lograr [ingresar datos de identificación y de domicilio] (1.1.1.1.1.1.2.1.1)}, y con el fenómeno \textbf{3a: el cliente presenta datos de identificación y de domicilio}.

Los pasos 2 y 3 se corresponden con el objetivo \textbf{lograr [validar usuario y contraseña] (1.1.1.1.1.1.2.2.4)}.

El paso 5 responde al objetivo \textbf{lograr [evaluar estado financiero] (1.1.1.1.1.1.2.2.2)}, y se corresponde con el fenómeno \textbf{3d: sistema evalúa estado financiero a través de financiera}.

El paso 7 responde al objetivo \textbf{lograr [validar domicilio] (1.1.1.1.1.1.2.2.1)}, y al fenómeno \textbf{3c: el sistema valida domicilio con el correo argentino}

El paso 8 responde al objetivo \textbf{lograr [presentar datos de pago, si quiere hacer pagos online] (1.1.1.1.1.1.2.1.2)} y al evento \textbf{3b: cliente presenta datos de pago al sistema}.

Los pasos 10 y 11 responden al objetivo \textbf{lograr [validar mail] (1.1.1.1.1.1.2.2.5)} y al evento \textbf{3e: el sistema le indica al correo electrónico que envíe mail de confirmación al cliente.}

%
% - CASO DE USO: 2) CONFIRMANDO DIRECCION DE MAIL
%
\begin{casodeuso}
  \cutitle{Confirmando dirección de mail}
  \cuactors{Cliente}
  \cupre{Se ha enviado un mail de confirmación}
  \cupost{El usuario se confirma}
  \cucourse{
    1. El cliente ingresa al link de confirmación & \\
    2. El sistema marca al usuario como validado & 2.1 Si el link no es válido, se le informa al usuario. \\
    3. FIN CU & FIN CU\\
  }
  \culabel{confirmando-direccion-de-mail}
\end{casodeuso}

Este caso de uso se corresponde con el objetivo \textbf{lograr [validar mail] (1.1.1.1.1.1.2.2.5)}, y con el evento al evento \textbf{3e: el sistema le indica al correo electrónico que envíe mail de confirmación al cliente}.

El paso 2 se corresponde con el objetivo \textbf{lograr [añadir cliente] (1.1.1.1.1.1.2.3)} y con el evento \textbf{3f: el sistema añade al cliente}.

%
% - CASO DE USO: 3) EVALUANDO ESTADO FINANCIERO
%
\begin{casodeuso}
  \cutitle{Evaluando estado financiero}
  \cuactors{Financiera}
  \cupre{}
  \cupost{}
  \cucourse{
    1. El sistema envía request a la API de estado financiero con el DNI del cliente & \\
    2. El sistema parsea la respuesta de la API & \\
  }
  \culabel{evaluado-estado-financiero}
\end{casodeuso}

Este caso de uso responde al objetivo \textbf{lograr [evaluar estado financiero] (1.1.1.1.1.1.2.2.2)}, y con el fenómeno \textbf{3e: sistema evalúa estado financiero a través de financiera}.

%
% - CASO DE USO: 4) ENVIANDO MAIL
%
\begin{casodeuso}
  \cutitle{Enviando mail}
  \cuactors{Correo electrónico}
  \cupre{El sistema definió un mensaje para un cliente}
  \cupost{Se ha delegado el envío de un correo electrónico}
  \cucourse{
    1. El sistema define el asunto, cuerpo y destinatario (cliente) del nuevo mensaje. & \\
    2. El sistema pide al servidor de correo electrónico enviar el mensaje. & \\
    3. El servidor de correo electrónico confirma el encolamiento del mensaje. & \\
  }
  \culabel{enviando-mail}
\end{casodeuso}

Este caso de uso se corresponde con el fenómeno \textbf{8e: el sistema de correo electrónico envía mail al cliente}. 

%
% - CASO DE USO: 5) AUTENTICANDOSE
%
\begin{casodeuso}
  \cutitle{Autenticándose}
  \cuactors{Cliente}
  \cupre{}
  \cupost{El cliente se encuentra autenticado}
  \cucourse{
    1. El cliente ingresa el usuario y la contraseña & \\
    2. El sistema verifica los datos ingresados por el cliente & \\
    3. El cliente es redirigido al portal de bienvenida & 3.1 Si los datos son inválidos, se muestra un mensaje de error \\
    4. FIN CU & 4.1 FIN CU\\
  }
  \culabel{autenticandose}
\end{casodeuso}

Este caso de uso se corresponde con el evento \textbf{4a) el cliente ingresa credenciales en el sistema}, y con el objetivo \textbf{lograr [cliente se autentica de forma segura] (1.1.1.1.1.2)}, y con sus subobjetivos.

%
% - CASO DE USO: 6) LISTANDO PRODUCTOS DISPONIBLES
%
\begin{casodeuso}
  \cutitle{Listando productos disponibles}
  \cuactors{Cliente}
  \cupre{El cliente está autenticado}
  \cupost{Se le envió al cliente un listado personalizado de productos}
  \cucourse{
    1. El cliente ingresa al listado de productos & \\
    2. El sistema obtiene los productos que están en stock & \\
    3. El sistema obtiene las recomendaciones para el usuario & \\
    4. El sistema muestra los productos y las recomendaciones que están en stock & \\
    5. FIN CU & \\
  }
  \culabel{listando-productos-disponibles}
\end{casodeuso}

Los pasos 1, 2 y 4 se corresponden con el objetivo \textbf{lograr [mostrar stock disponible] (1.1.1.1.2.1)} y con el fenómeno \textbf{4b: el sistema muestra stock disponible al cliente}.

Los pasos 3 y 4 se corresponden con el objetivo \textbf{lograr [mostrar recomendaciones] (1.1.1.1.2.2)} y con el fenómeno \textbf{4c: el sistema muestra recomendaciones al cliente.}

%
% - CASO DE USO: 7) AGREGANDO PRODUCTO AL CARRITO
%
\begin{casodeuso}
  \cutitle{Agregando producto al carrito}
  \cuactors{Cliente}
  \cupre{El cliente tiene el listado de productos}
  \cupost{El producto es agregado al carrito}
  \cucourse{
    1. El cliente hace click sobre el producto & \\
    2. El sistema muestra un dropdown con la cantidad de unidades que están disponibles en ese momento & \\
    3. El cliente elige la cantidad & \\
    4. El sistema agrega el producto al carrito y calcula el monto total  & \\
    5. FIN CU & \\
  }
  \culabel{agregado-producto-al-carrito}
\end{casodeuso}

Este caso de uso se corresponde con el objetivo \textbf{lograr [seleccionar mercadería] (1.1.1.1.2.3)} y con el fenómeno \textbf{4d: el cliente selecciona mercadería en el sistema}. El paso 2 se corresponde con el objetivo \textbf{lograr [mostrar stock disponible] (1.1.1.1.2.1)}.

%
% - CASO DE USO: 8) CONFIRMANDO COMPRA
%
\begin{casodeuso}
  \cutitle{Confirmando compra}
  \cuactors{Cliente}
  \cupre{El cliente tiene un carrito armado}
  \cupost{El cliente tiene una compra reservada y confirmada}
  \cucourse{
    1. El sistema ratifica la disponibilidad de stock para cada producto, y los reserva para el cliente; el carrito se encuentra reservado &  1.1 Si algún producto ya no tiene disponibilidad, se resta del carrito y se le informa al usuario; vuelve al paso 1. \\
    2. Incluye CU \ref{cu:calculando-costo-de-envio}: Calculando costo de envío & \\
    3. El sistema informa del costo total de la compra, incluyendo el envío. & \\
    4. Incluye CU \ref{cu:acordando-fecha-de-entrega}: Acordando fecha de entrega & \\
    5. El sistema determina si el cliente tiene autorizado el pago contraentrega, y presenta los métodos de pagos disponibles & \\
    6. El cliente aprueba el costo y las fechas informadas, e indica el método de pago deseado & \\
    7. Si el pago es online: Es extendido por CU \ref{cu:pagando-online}: Pagando online. & 7.1 Si el pago es online: si el pago no puede ser efectuado, vuelve a paso 7. \\
    8. El pedido es confirmado & 8.1 Si el pedido no pudo ser confirmado luego de 10 minutos, los productos son reingresados a stock, y el carrito deja de estar reservado, vuelve a paso 1 \\
    9. FI CU & \\
  }
  \culabel{confirmando-compra}
\end{casodeuso}

Este caso de uso responde al objetivo \textbf{lograr [confirmar carrito] (1.1.1.1.2.4)}, y al fenómeno \textbf{5a: el cliente confirma el carrito al sistema}.

Los pasos 1 y 1.1, se corresponden con el objetivo \textbf{lograr [si al confirmar carrito se acabó el stock de algún producto, se muestra mensaje de error y se quita producto del carrito, y se vuelve al estado previo a la confirmación] (1.1.1.1.2.5)} y con los fenómenos \textbf{5b: sistema ratifica al cliente stock disponible} y \textbf{5c: sistema muestra mensaje de error de producto no disponible al cliente}.

El paso 2 se corresponde con el fenómeno \textbf{5d: sistema obtiene el costo del envío de la API de Logística} y con el objetivo \textbf{lograr [calcular costo de envío] (1.1.1.1.4.4)}.

El paso 3 se corresponde con el fenómeno \textbf{5e: sistema informa el costo de la compra y el envío al cliente}, y con el objetivo \textbf{lograr [informar costo total] (1.1.1.1.4.5}.

El paso 4 se corresponde con los fenómenos \textbf{5f, 5g y 5h}, y con el objetivo \textbf{lograr [acordar fecha de entrega] (1.1.1.1.3)} y sus subobjetivos.

El paso 5 se corresponde con el evento \textbf{5i: sistema presenta los métodos de pago disponibles al cliente} y con los objetivos \textbf{lograr [evaluar si permitir contraentrega] (1.1.1.1.4.1)} y \textbf{lograr [informar métodos de pago disponibles] (1.1.1.1.4.2)}.

El paso 6 se corresponde con el evento \textbf{5j: el cliente elige el método de pago} y con el objetivo \textbf{lograr [elegir forma de pago] (1.1.1.1.4.3)}.

Los fenómenos \textbf{5k, 5l, 5m, 5n, 5o y 5p} se corresponden con el paso 7, y con el objetivo \textbf{lograr [si la forma de pago es online, se cobra online] (1.1.1.1.5)}. 

El paso 7.1 se corresponde con el evento \textbf{5q: El sistema informa de error de pago al cliente} y está relacionado con el objetivo \textbf{lograr [si la forma de pago es online, se cobra online] (1.1.1.1.5)}.

El paso 8.1 se corresponde con el evento \textbf{5r: sistema muestra error de timeout de carrito al cliente}, que se corresponde con \textbf{lograr [si el cliente no confirma el pedido en un intervalo de timeout, el pedido vuelve a estado de carrito] (1.1.1.1.6)}.

%
% - CASO DE USO: 9) CALCULAR COSTO DE ENVIO
%
\begin{casodeuso}
  \cutitle{Calculando costo de envío}
  \cuactors{API de Logística}
  \cupre{El domicilio fue previamente validado.}
  \cupost{}
  \cucourse{
    1. El sistema le consulta a la API de logística por el costo de envío hacia el domicilio de un cliente & \\
    2. La API le devuelve al sistema el costo de envío asociado a ese domicilio. & \\
    3. FIN CU & \\
  }
  \culabel{calculando-costo-de-envio}
\end{casodeuso}

%
% - CASO DE USO: 10) ACORDANDO FECHA DE ENTREGA
%
\begin{casodeuso}
  \cutitle{Acordando fecha de entrega}
  \cuactors{Cliente, API de Logística}
  \cupre{El cliente tiene un carrito reservado}
  \cupost{El pedido tiene fecha tentativa de entrega}
  \cucourse{
    1. El sistema pregunta próximas fechas libres a la API de logística & \\
    2. El sistema presenta las posibles fechas al cliente & \\
    3. El cliente elige la fecha deseada & \\
    4. FIN CU & \\
  }
  \culabel{acordando-fecha-de-entrega}
\end{casodeuso}

El paso 1 se corresponde con el evento \textbf{7f: sistema pregunta próximas fechas libres a la API de logística}
El paso 2 se corresponde con el evento \textbf{7g: el sistema presenta las posibles fechas de entrega al cliente}
El paso 3 se corresponde con el evento \textbf{7h: el cliente indica la fecha de entrega deseada al sistema}

%
% - CASO DE USO: 11) PAGANDO ONLINE
%
\begin{casodeuso}
  \cutitle{Pagando online}
  \cuactors{Cliente, Agente de Cobro}
  \cupre{El cliente eligió el método de pago online}
  \cupost{El pago del cliente fue acreditado}
  \cucourse{
    1. Si el cliente lo desea, es extendido por CU \ref{cu:agregando-datos-de-pago-online}: agregando datos de pago online & \\
    2. El sistema muestra datos de pago asociados al cliente & \\
    3. El cliente indica método y datos de pago & 3.1 Si el cliente no posee datos de pago, ir a paso 1 \\
    4. El sistema abre una ventana del agente de cobro con los datos de la transacción. & \\
    5. El cliente realiza la operación a través del agente de cobro, generando un token comprobante del pago. & 5.1 Si el agente de cobro rechaza el pago, ir al paso 1.\\
    6. El sistema recibe el comprobante de pago, y lo verifica contra el agente de pago. & 6.1 Si el token de pago es inválido, informar al usuario, e ir al paso 1 \\
    7. El sistema prepara un mail para el cliente, adjuntando un comprobante de pago de la operación.
    8. Incluye caso de uso \ref{cu:enviando-mail}: enviando mail & \\
    9. FIN CU & \\
  }
  \culabel{pagando-online}
\end{casodeuso}

% 7k: El sistema muestra datos de pago asociados al cliente
% 7l: El cliente indica dato de pago
% 7m: El sistema redirige al cliente a la ventana del Agente de Cobro
% 7n: El cliente autoriza la operación en la ventana del Agente de Cobro
% 7o: El agente de cobro envía estado y comprobante al sistema
% 7p: El sistema pide mail de comprobante de pago al Correo Electrónico

%
% - CASO DE USO: 12) AGREGANDO DATOS DE PAGO ONLINE
%
\begin{casodeuso}
  \cutitle{Agregando datos de pago online}
  \cuactors{Cliente}
  \cupre{El cliente está autenticado}
  \cupost{El cliente posee un nuevo dato de pago asociado a su cuenta}
  \cucourse{
    1. El cliente elige el método de pago online de entre las opciones disponibles & \\
    2. Según el método de pago elegido, el cliente ingresa los datos de autenticación solicitados. & \\
    3. Incluye: validando datos de pago & \\
    4. El sistema asocia los datos de pago a la cuenta del cliente & 4.1 Si los datos de pago son inválidos regresa a paso 1. \\
    5. FIN CU & \\
  }
  \culabel{agregando-datos-de-pago-online}
\end{casodeuso}

%
% - CASO DE USO: 13) CANCELANDO PEDIDO
%
\begin{casodeuso}
  \cutitle{Cancelando pedido}
  \cuactors{Cliente}
  \cupre{El cliente tiene un pedido sin armar en depósito.}
  \cupost{El pedido es cancelado.}
  \cucourse{
    1. Si el carrito está confirmado, la reserva de productos se anula, y los mismos se reingresan a stock. & \\
    2. Si el pedido ya fue pagado de forma online, es extendido por CU \ref{cu:reintegrado-dinero}: Reintegrando dinero & \\
    3. El pedido es cancelado. & \\
    4. FIN CU & \\
  }
  \culabel{cancelando-pedido}
\end{casodeuso}

%
% - CASO DE USO: 14) MODIFICANDO PEDIDO
%
\begin{casodeuso}
  \cutitle{Modificando pedido}
  \cuactors{Cliente}
  \cupre{El cliente tiene un pedido sin armar en depósito.}
  \cupost{El pedido es cancelado, y un nuevo pedido con las modificaciones es creado.}
  \cucourse{
    1. El cliente quita o agrega los productos que desee, siempre y cuando haya disponibilidad de stock. & \\
    2. Si el carrito está sin confirmar, se registra la modificación & \\
    3. Si el carrito fue confirmado, se cancela el pedido anterior: es extendido por CU \ref{cu:cancelando-pedido} Cancelando Pedido. & \\
    4. Si el carrito fue cancelado, se genera un nuevo pedido con los datos modificados. & \\ 
    5. Si un nuevo pedido fue generado, es extendido por CU \ref{cu:confirmando-compra}: Confirmando compra. & \\
    6. FIN CU & \\
  }
  \culabel{modificando-pedido}
\end{casodeuso}

%
% - CASO DE USO: 15) VALIDANDO DATOS DE PAGO
%
\begin{casodeuso}
  \cutitle{Validando datos de pago}
  \cuactors{Agente de Cobro}
  \cupre{El cliente ingresó los datos de pago}
  \cupost{Los datos de pago fueron validados}
  \cucourse{
    1. El sistema envía los datos de pago del cliente al Agente de Cobro a través de una API.& \\
    2. El Agente de Cobro informa sobre la validez de los datos de pago. & \\
    3. FIN CU & 2.1 Si los datos de pago no son válidos, el sistema los marca como inválidos, FIN CU\\
  }
  \culabel{validando-datos-de-pago}
\end{casodeuso}

%
% - CASO DE USO: 16) REINTEGRANDO DINERO
%
\begin{casodeuso}
  \cutitle{Reintegrando dinero}
  \cuactors{Agente de cobro}
  \cupre{El cliente canceló un pedido}
  \cupost{El dinero correspondiente al pedido fue reintegrado al cliente}
  \cucourse{
    1. El sistema contacta al agente de cobro, solicitando la anulación de las operaciones correspondientes al pago del pedido. & \\
    2. El agente de cobro anula las operaciones de pago solicitadas, y entrega un número de operación y un comprobante de anulación para cada una de ellas. & \\
    3. El sistema prepara un mensaje para el cliente, informando que el pedido fue anulado, adjuntando los comprobantes de devolución, y lo devuelve al cliente por pantalla. & 3.1 Si el pago no puede ser anulado, se le informa de la situación al cliente, brindándole los números de operación correspondiente.\\
    4. Utilizando el mensaje anterior, incluye caso de uso \ref{cu:enviando-mail}: Enviando mail. & \\
    5. FIN CU & \\
  }
  \culabel{reintegrado-dinero}
\end{casodeuso}

%
% - CASO DE USO: 17) VALIDANDO DOMICILIO
%
\begin{casodeuso}
  \cutitle{Validando domicilio}
  \cuactors{API de Correo Argentino}
  \cupre{El cliente ingresó los datos de su domicilio}
  \cupost{Los datos de domicilio del cliente fueron validados}
  \cucourse{
    1. El sistema envía los datos de domicilio del cliente mediante la API del Correo Argentino & \\
    2. La API del Correo Argentino responde con un mensaje informando la validez del domicilio enviado & \\
    3. FIN CU & \\
  }
  \culabel{validando-domicilio}
\end{casodeuso}

Este caso de uso responde al objetivo \textbf{lograr [validar domicilio] (1.1.1.1.1.1.2.2.1)} y al fenómeno \textbf{3c: el sistema valida domicilio con el correo argentino}.

%
% - CASO DE USO: 18) PREPARANDO PEDIDO
%
\begin{casodeuso}
  \cutitle{Preparando pedido}
  \cuactors{Depósito, Logística}
  \cupre{El cliente tiene un pedido confirmado}
  \cupost{El cliente tiene un pedido preparado}
  \cucourse{
    1. El sistema brinda el listado de  productos a preparar, junto con el domicilio y la fecha y hora de entrega & \\
    2. Un operario del depósito marca el pedido como preparado & \\
    3. FIN CU & \\
  }
  \culabel{preparando-pedido}
\end{casodeuso}

%
% - CASO DE USO: 19) REGISTRANDO ENTREGA SATISFACTORIA
%
\begin{casodeuso}
  \cutitle{Registrando entrega satisfactoria}
  \cuactors{Logística}
  \cupre{Se ha realizado una entrega satisfactoria}
  \cupost{La entrega fue registrada en el sistema}
  \cucourse{
    1. Logística registra la entrega al cliente satisfactoria & \\
    2. Si el pago fue contraentrega, logística carga los datos de recepción del dinero & \\
    3. FIN CU & \\
  }
  \culabel{registrando-entrega-satisfactoria}
\end{casodeuso}

%
% - CASO DE USO: 20) REGISTRANDO ENTREGA FALLIDA
%
\begin{casodeuso}
  \cutitle{Registrado entrega fallida}
  \cuactors{Depósito}
  \cupre{La entrega del pedido fue fallida}
  \cupost{El pedido está anulado y los productos aprobados fueron reingresados a stock}
  \cucourse{
    1. La mercadería en buen estado es reingresada a stock & \\
    2. Depósito carga la falta del cliente y el costo generado a la empresa & \\
    3. El pedido es anulado. & \\
    4. El sistema genera un mensaje conteniendo una invitación a rehacer el pedido, y un link hacia una orden de compra con el mismo carrito del pedido anulado & \\
    5. Incluye CU \ref{cu:enviando-mail}: Enviando mail. & \\
    6. FIN CU & \\
  }
  \culabel{registrando-entrega-fallida}
\end{casodeuso}

%
% - CASO DE USO: 21) CARGANDO PEDIDO A PROVEEDOR
%
\begin{casodeuso}
  \cutitle{Cargando pedido a proveedor}
  \cuactors{Departamento de Stock}
  \cupre{Se hizo un pedido a un proveedor}
  \cupost{El pedido está cargado}
  \cucourse{
    1. El Departamento de Stock carga la información del pedido y su comprobante al sistema. & \\
    2. FIN CU & \\
  }
  \culabel{cargando-pedido-a-proveedor}
\end{casodeuso}

%
% - CASO DE USO: 22) CARGANDO RECEPCIÓN DE PEDIDO A PROVEEDOR
%
\begin{casodeuso}
  \cutitle{Cargando recepción de pedido a proveedor}
  \cuactors{Depósito}
  \cupre{Llegó un pedido al depósito}
  \cupost{Los nuevos productos son ingresados a stock}
  \cucourse{
    1. Depósito registra el ingreso. & \\
    2. El sistema actualiza el stock de los nuevos productos & \\
    3. FIN CU & \\
  }
  \culabel{cargando-recepcion-de-pedido-a-proveedor}
\end{casodeuso}

%
% - CASO DE USO: 23) INFORMANDO PREPARACIÓN DE STOCK PARA REPOSICIÓN
%
\begin{casodeuso}
  \cutitle{Informando preparación de stock para reposición}
  \cuactors{Depósito}
  \cupre{Hay un pedido de reposición de la sucursal}
  \cupost{Los productos requeridos son restados del stock}
  \cucourse{
    1. Depósito informa al sistema que el pedido de reposición se encuentra en preparación. & \\
    2. El sistema resta del stock las unidades correspondientes al pedido. & \\
    3. El sistema le brinda una respuesta al depósito, ofreciendo la descarga de una planilla que contiene un sumario de las unidades que deberán ser enviadas a la sucursal, junto con la información interna que permita agilizar el proceso de preparación (localización dentro del depósito, números de empaque, etcétera). & \\
    4. El  Depósito informa la correcta preparación del pedido. & \\ 
    5. El sistema marca el pedido preparado, y pone a disposición del Depósito una una hoja de ruta / remito de traslado, para su impresión y futura utilización. & \\
    6. FIN CU & \\
  }
  \culabel{informando-preparacion-de-stock-para-reposicion}
\end{casodeuso}

%
% - CASO DE USO: 24) REGISTRANDO ENTREGA DE REPOSICIÓN
%
\begin{casodeuso}
  \cutitle{Registrando entrega de reposición}
  \cuactors{Sucursal}
  \cupre{Un envío de reposición a sucursal fue entregado correctamente}
  \cupost{La llegada del envío es registrada}
  \cucourse{
  1. La sucursal ingresa al sistema, y marca el pedido de reposición como entregado. & \\
  2. FIN CU. & \\
}
  \culabel{registrando-entrega-de-reposicion}
\end{casodeuso}

%
% - CASO DE USO: 25) DEFINIENDO UMBRAL DE REDITUABILIDAD
%
\begin{casodeuso}
  \cutitle{Definiendo umbral de redituabilidad}
  \cuactors{Administrador}
  \cupre{-}
  \cupost{Se redefine el umbral de redituabilidad}
  \cucourse{
    1. El administrador ingresa el nuevo umbral de redituabilidad & \\
    2. El sistema guarda el nuevo umbral & \\
    3. FIN CU. & \\
  }
  \culabel{definiendo-umbral-de-redituabilidad}
\end{casodeuso}

%
% - CASO DE USO: 26) DANDO DE ALTA PRODUCTO
%
\begin{casodeuso}
  \cutitle{Dando de alta producto}
  \cuactors{Administrador}
  \cupre{-}
  \cupost{El producto solicitado es dado de alta}
  \cucourse{
    1. El administrador ingresa al ABM provisto por el sistema & \\
    2. El administrador ingresa el producto & \\
    3. El sistema da de alta el producto y comunica la operación & \\
    4. FIN CU. & \\
  }
  \culabel{dando-de-alta-producto}
\end{casodeuso}

%
% - CASO DE USO: 27) DANDO DE BAJA PRODUCTO
%
\begin{casodeuso}
  \cutitle{Dando de baja producto}
  \cuactors{Administrador}
  \cupre{-}
  \cupost{El producto solicitado es dado de baja}
  \cucourse{
    1. El administrador ingresa al ABM provisto por el sistema & \\
    2. El administrador ingresa el producto & \\
    3. El sistema da de baja el producto y comunica la operación & \\
    4. FIN CU. & \\
  }
  \culabel{dando-de-baja-producto}
\end{casodeuso}

%
% - CASO DE USO: 28) MODIFICANDO PRODUCTO
%
\begin{casodeuso}
  \cutitle{Modificando producto}
  \cuactors{Administrador}
  \cupre{-}
  \cupost{El producto solicitado es modificado}
  \cucourse{
    1. El administrador ingresa al ABM provisto por el sistema & \\
    2. El administrador ingresa el producto y la modificación & \\
    3. El sistema realiza los cambios solicitados y confirma la operación & \\
    4. FIN CU. & \\
  }
  \culabel{modificando-producto}
\end{casodeuso}

%
% - CASO DE USO: 29) OBTENIENDO ESTADISTICAS
%
\begin{casodeuso}
  \cutitle{Obteniendo estadísticas}
  \cuactors{Administrador}
  \cupre{-}
  \cupost{Se envían las estadísticas al administrador}
  \cucourse{
    1. El administrador solicita las estadísticas a través del sistema & \\
    2. El sistema genera las estadísticas de venta de cada producto y compras de cada usuario, y las prepara para su adecuada visualización & \\
    3. El administrador descarga las estadísticas a través del sistema & \\
    4. Si administrador desea modificar algún producto, es extendido por CU \ref{cu:modificando-producto}: Modificando producto. & \\
    5. FIN CU. & \\
  }
  \culabel{obteniendo-estadisticas}
\end{casodeuso}

%
% - CASO DE USO: 30) ENCARGANDO REPOSICIÓN
%
\begin{casodeuso}
  \cutitle{Encargando reposición}
  \cuactors{Sucursal}
  \cupre{La sucursal se conecta desde una terminal especial pre-autenticada.}
  \cupost{Los productos deseados fueron encargados}
  \cucourse{
    1. La sucursal ingresa el listado de productos que desea encargar. & \\
    2. El sistema guarda el pedido como pendiente de preparación. & \\
    3. El sistema le confirma a la sucursal que el pedido fue encargado. & \\
    4. FIN CU. & \\
  }
  \culabel{encargado-reposicion}
\end{casodeuso}

%
% - CASO DE USO: 31) CARGANDO DATOS SEMANALES DE VENTA
%
\begin{casodeuso}
  \cutitle{Cargando datos semanales de venta}
  \cuactors{Sucursal}
  \cupre{La sucursal tiene datos de venta para informar}
  \cupost{El sistema contiene datos de venta actualizados}
  \cucourse{
    1. La sucursal carga los datos de venta de la semana a través de una interfaz sencilla, por ejemplo una planilla de cálculos. & \\
    2. El sistema procesa los datos de venta, y los integra a su base de estadísticas & \\
    3. FIN CU. & \\
  }
  \culabel{cargando-datos-semanales-de-venta}
\end{casodeuso}

%
% - CASO DE USO: x) ENVIANDO ALARMA DE BAJO STOCK
%
%\begin{casodeuso}
%\cutitle{21) Enviando alarma de bajo stock}
%\cuactors{Correo electronico}
%\cupre{Existe al menos un producto con stock menor al límite estipulado}
%\cupost{Se avisa de la falta de stock al Departamento de Stock}
%\cucourse{
%1. El sistema prepara una lista de todos los productos por debajo del límite estipulado y el stock necesario para reestablecerlos & \\
%2. El sistema de correo electrónico envía un mail con los datos al Departamento de Stock & \\
%3. FIN CU & \\
%}
%\end{casodeuso}
