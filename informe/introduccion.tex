%--
%  Introducción: acá debería figurar todo lo necesario como para que un lector
%  no iniciado en el tema entienda el propósito general del sistema que van a
%  describir. Puede citar algunos fragmentos del enunciado si lo consideran
%  necesario, pero NO DEBE SER una copia del enunciado.
%--

\section{Introducción}

El presente Trabajo Práctico es una extensión del TP1: Mes\%. Utilizando como base la temática y los modelos descriptos en el anterior, aplicamos diversas técnicas para documentar distintos aspectos del proyecto. Realizamos un Modelo Conceptual, abarcando todos los aspectos del sistema, para describir las entidades que intervienen en el mismo, sus atributos, y la semántica de sus relaciones, y un Modelo de Casos de Uso para describir las operaciones desde el punto de vista de los distintos agentes, representadas funcionalmente a través de un listado de acciones provistas, producidas o transmitidas por la máquina a través de la interfaz.

Para poder representar algunos aspectos más complejos decidimos utilizar otros modelos que nos permiten modelar los comportamientos de los agentes. En particular, usamos Diagramas de Actividad, para representar aquellas acciones o actividades que si bien no forman parte intrínseca de la máquina, influyen en la operatoria del sistema. Más concretamente, mediante la utilización de este modelo, pudimos caracterizar el flujo de las operaciones descriptas en Modelo de Casos de Uso, así como su interacción con agentes externos al sistema.

Finalmente, mediante el Modelo de Máquinas de Estado Finitas (FSM) pudimos representar las transiciones de estados que puede sufrir un agente, de acuerdo a los eventos que se disparan en el sistema. También nos sirvió para mostrar los fenómenos de sincronización, pudiendo darle simultaneidad arbitraria a los cambios de agentes partícipes de un mismo evento.

Tuvimos que tomar algunas decisiones sobre los o-refinamientos planteados en el Diagrama de Objetivos del Trabajo Práctico anterior. Con respecto a la forma de efectuar el registro, se optó por la modalidad online, ya que se trata de una metodología establecida hace tiempo en el mercado que ya es parte de la experiencia de usuario esperada por los clientes. También se decidió delegar la decisión de permitir contraentrega a un cliente a la evaluación de su nivel de redituabilidad, ya que se consideró este un mecanismo de control que no requiere participación activa de las partes implicadas, y que resulta eficaz para prevenir las pérdidas asociadas a envíos fallidos.
