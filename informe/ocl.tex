OCL}

\begin{listocl}

%
% - OCL: 1
%
\begin{itemocl}{La redituabilidad de una unidad es su precio de venta menos su precio de lista.}
Context: Unidad
Inv: self.redituabilidad = self.precioDeVenta - self.precioDeLista
  \end{itemocl}


%
% - OCL: 2
%
  \begin{itemocl}{La redituabilidad de un pedido es la ganancia menos el costo.}
Context: Pedido

ganancia(p) = if p.estadoPedido == Concretado
	   then p.unidades->collect(redituabilidad)->sum()
	   else 0
	   endif
costo(p) = if p.envio->size() == 1
	then p.envio.costo
	else 0
	endif
	+ if p.envio->size() == 1 and p.envio.oclIsTypeOf(EnvioFallido)
	then p.envio.costoMercaderiaMalograda
	else 0
	endif
Inv: self.redituabilidad = ganancia(self) - costo(self)
  \end{itemocl}


%
% - OCL: 3
%
  \begin{itemocl}{Se le permite la contraentrega si y solo si la redituabilidad del cliente es mayor al umbral de redituabilidad.}
Context: Cliente Particular

redituabilidadCliente(c) = c.pedidos->collect(redituabilidad)->sum()
Inv: self.permitidoContraentrega = redituabilidadCliente(self) > GLOBAL_VARIABLE(umbralRedituabilidad)
  \end{itemocl}


%
% - OCL: 4
%
  \begin{itemocl}{Un cliente no puede tener dos pedidos vigentes.}
Context: Cliente Particular
Inv: self.pedidos->filter(x | {entregado, anulado, cancelado}->excludes(x.estado)).size() < 2
  \end{itemocl}

%
% - OCL: 5
%
  \begin{itemocl}{Un Producto está bajo en stock en un Depósito si y sólo sí la cantidad de unidades de dicho producto en el Depósito está por debajo del umbral mínimo de dicho producto.}
Context: Producto
Inv: Deposito.allInstances()->forall(deposito | 
    (self.estaEnBaja->filter(deposito2 | deposito = deposito2).size() = 1) =
    (deposito.tengo->filter(unidad | unidad.soy = self).size() < self.umbralMinimo)
)
  \end{itemocl}

%
% - OCL: 6
%
  \begin{itemocl}{Para cada pedido, $fechaCreacion < fechaDespacho < fechaEntrega$}
Context: Envio
Inv: self.sobrePedido.fechaCreacion < self.fechaDespacho < self.sobrePedido.fechaEntrega
  \end{itemocl}

%
% - OCL: 7
%
  \begin{itemocl}{
      Una unidad está en solo uno de los siguientes lugares:
          pedidoAProveedor
          depósito
          depósito y en pedido en estado enCarrito, anulado, cancelado
          en pedido confirmado
    }
Context: Unidad
Inv:
(self.enPedidoAProovedor->notEmpty() and self.enDeposito->isEmpty() and self.enPedido->isEmpty())
or
(self.enDeposito->notEmpty() and self.enPedidoAProovedor->isEmpty() and self.enPedido->isEmpty())
or
(self.enDeposito->notEmpty() and self.enPedidoAProovedor->isEmpty() and self.enPedido->notEmpty() and {enCarrito, cancelado, anulado}->includes(self.enPedido.estado))
or
(self.enDeposito->isEmpty() and self.enPedidoAProovedor->isEmpty() and self.enPedido->filter(p | {enCarrito, cancelado, anulado}->excludes(p.estado)).size() == 1)
  \end{itemocl}

%
% - OCL: 8
%
  \begin{itemocl}{
      Un cliente C tiene un producto recomendado P si y solo si:
          C nunca compro P y
          P pertenece a la lista de 10 más comprados de algún cliente parecido a C
    }
Context: Cliente Particular
  edadParecida(cliente1, cliente2) = |cliente1.edad - cliente2.edad| < 4
  
  productosComprados(cliente) = cliente.hicePedido.incluyo.soy
  
  cantUnidadesCompradas(producto, cliente) = productosComprados(cliente)->filter(p | p = producto)->size()

  cantUnidadesCompradasOrdenadas(cliente) = productosComprados(cliente)->asSet()->sortedBy(p | cantUnidadesCompradas(p, cliente))->reverse()

  diezMasComprados(cliente) = cantUnidadesCompradasOrdenadas(cliente)->subSequence(1, 10)

  clienteParecido(cliente1, cliente2) = edadParecida(cliente1, cliente2) and intersection(diezMasComprados(cliente1), diezMasComprados(c2))->size() >= 5

  Inv: self.meRecomiendan->forAll(producto | productosComprados(self)->excludes(producto) 
      and ClienteParticular.allInstances()->exists(c | clienteParecido(self, c) and diezMasComprados(c)->includes(p))
  )
  \end{itemocl}

%
% - OCL: 9
%
  \begin{itemocl}{Comprobantes deben tener asociados datos de pago registrados por el cliente.
Observación: la tarjeta puede tener como titular a un familiar, pero quien la registra es el cliente.}
    Context: Comprobante de tarjeta de credito
    Inv: self.tarjeta.asociadoA = self.pagoDePedido.hechoPor
    Context: Comprobante PayPal
    Inv: self.cuenta.asociadoA = self.pagoDePedido.hechoPor
  \end{itemocl}
\end{listocl}
