\subsection{OCL}

\begin{listocl}
  \begin{itemocl}{La redituabilidad de una unidad es su precio de venta menos su precio de lista.}
Context: Unidad
Inv: self.redituabilidad = self.precioDeVenta - self.precioDeLista
  \end{itemocl}

  \begin{itemocl}{La redituabilidad de un pedido es la ganancia menos el costo.}
Context: Pedido

ganancia = if self.estadoPedido == Concretado
	   then self.unidades->collect(precioDeVenta-precioDeLista)->sum()
	   else 0
	   endif
costo = if self.envio
	then self.envio.costo
	else 0
	endif
	+ if self.envio.oclIsTypeOf(EnvioFallido)
	then self.envio.costoMercaderiaRechazada
	else 0
	endif
Inv: self.redituabilidad = ganancia - costo
  \end{itemocl}

  \begin{itemocl}{Se le permite la contraentrega si la redituabilidad del cliente es mayor al umbral de redituabilidad.}
Context: Cliente Particular

redituabilidadCliente = self.pedidos->collect(redituabilidad)->sum()
Inv: iff(permitidoContraentrega, redituabilidadCliente > umbralRedituabilidad
  
  \end{itemocl}

  \begin{itemocl}{Un cliente no puede tener dos pedidos vigentes.}
Context: Cliente Particular
Inv: self.pedidos->filter(x | x.estado != entregado and x.estado != anulado and x.estado != cancelado).size() < 2
  \end{itemocl}

  \begin{itemocl}{Un Producto está bajo en stock en un Depósito sii la cantidad de unidades de dicho producto en el Depósito está por debajo del umbral mínimo de dicho producto.}
Context: Producto
Inv: self.estaEnBajaEn->forall(d | d.tengo->filter(u | u.soy = self).size() < self.umbralMínimo)
  \end{itemocl}

\end{listocl}
