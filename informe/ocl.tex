\subsection{OCL}

\begin{listocl}
  \begin{itemocl}
   \ocldescription{La redituabilidad de una unidad es su precio de venta menos su precio de lista.}
   \oclcontext{Unidad}
   \oclbody{self.redituabilidad = self.precioDeVenta - self.precioDeLista}
  \end{itemocl}

  \begin{itemocl}
    \ocldescription{La redituabilidad de un pedido es la ganancia menos el costo.}
    \oclcontext{Pedido}
    \oclbody{
      ganancia = if self.estadoPedido == Concretado
                      then self.unidades->collect(precioDeVenta-precioDeLista)->sum()
                      else 0
                      endif
      costo = if self.envio
              then self.envio.costo
              else 0
              endif
              + if self.envio.oclIsTypeOf(EnvioFallido)
              then self.envio.costoMercaderiaRechazada
              else 0
              endif
      Inv: self.redituabilidad = ganancia - costo
    }
  \end{itemocl}

  \begin{itemocl}
    \ocldescription{Se le permite la contraentrega si la redituabilidad del cliente es mayor al umbral de redituabilidad.}
    \oclcontext{Cliente Particular}
    \oclbody{
      redituabilidadCliente = self.pedidos->collect(redituabilidad)->sum()
      Inv: iff(permitidoContraentrega, redituabilidadCliente > umbralRedituabilidad
    }
  \end{itemocl}

  \begin{itemocl}
   \ocldescription{Un cliente no puede tener dos pedidos vigentes.}
   \oclcontext{Cliente Particular}
   \oclbody{self.pedidos->filter(x | x.estado != entregado and x.estado != anulado and x.estado != cancelado).size() < 2}
  \end{itemocl}

  \begin{itemocl}
   \ocldescription{Un Producto está bajo en stock en un Depósito sii la cantidad de unidades de dicho producto en el Depósito está por debajo del umbral mínimo de dicho producto.}
   \oclcontext{Producto}
   \oclbody{self.estaEnBajaEn->forall(d | d.tengo->filter(u | u.soy = self).size() < self.umbralMínimo)}
  \end{itemocl}

\end{listocl}
